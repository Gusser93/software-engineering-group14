\documentclass{article}
\usepackage[T1]{fontenc}
\usepackage[utf8]{inputenc}
\title{Review Document -- Requirements Document}
\author{David Klopp \and Christian Stricker \and Markus Vieth}
\date{\today}
%Requirement, Problem, Solution
\newcommand{\RPS}[5]{\subsection{#1 (#2)} \subsubsection*{Statement:} ''#3'' \subsubsection*{Problem:} #4 \subsection*{Lösung:} #5} 

\begin{document}
\maketitle
\cleardoublepage
\tableofcontents 
\newpage
\section{Introduction}

\section{User-Requirements}

\RPS{Mit Algorithmen "arbeiten"}{UR002}{%
Das System soll mit den Klassifikationsalgorithmen J48, Random Forest und SMO arbeiten können.%
}{%
Problem%
}{%
Lösung%
}

\RPS{Daten}{UR004}{%
Das System soll alle Daten in einer Datenbank verwalten..%
}{%
Problem%
}{%
Lösung%
}


\RPS{Ordnerstruktur}{UR005}{%
Die Datenbank soll eine Ordnerstruktur verwalten.%
}{%
Problem%
}{%
Lösung%
}


\RPS{Superadmin}{UR006}{%
Das System soll durch einen Superadmin verwaltet und gewartet werden.%
}{%
Problem%
}{%
Lösung%
}


\RPS{Trainingsdaten}{UR015}{%
Der Benutzer soll auf bestehende Modelle Trainingsdaten anwenden können, sofern er die Berechtigung dazu hat.%
}{%
Problem%
}{%
Lösung%
}



\newpage


\section*{System Requirements}

\section{Non-Functional Requirements}

\RPS{Nutzerdaten sollen "sicher" sein}{NFR010}{%
Alle Nutzerdaten sollen sicher sein und werden nicht außerhalb des Systems verwendet.%
}{%
Problem%
}{%
Lösung%
}


\RPS{Speicherort der Passwörter}{NFR013}{%
Passwörter der Benutzer sollen mit SHA-256 und einem Salt sicher verschlüsselt sein.%
}{%
Problem%
}{%
Lösung%
}


\section{Functional Requirements}

\RPS{Protokoll}{FR002}{%
Das System soll die übergebenen Datensätze auf dem Server verarbeiten.%
}{%
Das Functional Requirement ist ungenau und nicht messbar. Es wird weder genannt, in wie fern Datensätze verarbeitet werden, noch was mit ihnen passieren soll.%
}{%
Als User Requirement notieren oder streichen%
}


\RPS{Warteschlange}{FR007}{%
Es sollen standardmäßig zwei Modelle gleichzeitig berechnet werden können. Alle weiteren Berechnungen sind in einer Warteschlange. Der Superadmin soll einstellen können, wie viele gleichzeitig berechnet werden können.%
}{%
Die Art der Warteschlange ist nicht definiert.%
}{%
Die Warteschlange könnte als first in, first out festgelegt werden.%
}


\RPS{Ordner}{FR016}{%
Ein Ordner soll Paket heißen und Trainingsdatensatz, Testdatensatz, Modell und Ergebnisdatensätze enthalten. Pakete können geschachtelt werden.%
}{%
Der Begriff Ordner an dieser Stelle ist falsch/ungenau. %
}{%
Das Wort Ordner streichen.%
}


\RPS{"verschiedene Server"}{FR017}{%
Das System soll auf verschiedene Server zugreifen können.%
}{%
Die Formulierung ist mehrdeutig. Ein System soll auf einem Server laufen. Der Zugriff auf andere Datenbanken erfolgt über das REST-Interface.%
}{%
Genauer ausformulieren.%
}


\RPS{Sicherheitslücke}{FR020}{%
Algorithmen sollen von angemeldeten Benutzern der System-Bibliothek hinzugefügt werden können.%
}{%
Große Sicherheitslücke, da so Schadcode in das System gespielt werden kann.%
}{%
Der Administrator überprüft die Algorithmen vor der Integration in das System auf ihre Funktionalität und eventuelle Schadsoftware.%
}

\RPS{Sicherheitslücke}{FR021}{%
Modelle sollen als serialisiertes Java Objekt portierbar sein.%
}{%
Große Sicherheitslücke, da so Schadcode in das System gespielt werden kann.%
}{%
Modelle sind in einem anderem Format zu importieren oder durch einen Administrator zu prüfen.%
}


\RPS{Erweitertes GUI und erweiterte Funktionen}{FR019}{%
Der Superadmin soll eine erweiterte GUI erhalten, in der seine erweiterten Funktionen ausführbar sind.%
}{%
"`Erweitertes GUI"' und "`erweiterte Funktionen"' sind nicht weiter definiert.%
}{%
Als User Requirement aufnehmen oder genauer definieren.%
}


\end{document}

%%% Local Variables:
%%% mode: latex
%%% TeX-master: t
%%% End:
