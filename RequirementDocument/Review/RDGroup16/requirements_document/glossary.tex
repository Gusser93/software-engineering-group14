\newglossaryentry{ASCII}
{
	name=ASCII,
	description={Ist eine Zeichencodierung. (engl. American Standart Code for Information Interchange)}
}
\newglossaryentry{arff}
{	
	name=arff,
	description={Ist ein Datenformat, welches in \gls{ASCII}-Text geschrieben ist und eine Liste von Instanzen beschreibt, die dieselben Attribute teilen. \newline(Quelle: http://www.cs.waikato.ac.nz/ml/weka/arff.html 11.11.2015)}
}
\newglossaryentry{Bootstrap}
{
	name=Bootstrap,
	description={Ist ein System, welches \gls{HTML}, \gls{CSS} und \gls{Javascript}, jeweils auf Bildschirmgröße anpasst.}
}
\newglossaryentry{CSS}
{
name=Cascading Style Sheets,
description={Ist ein Sprache, mit der Gestaltungsanweisungen in \gls{HTML} erstellt werden können.
\newline (Quelle: de.wikipedia.org 11.11.2015)}
}
\newglossaryentry{csv}
{
name=csv,
description={Ist ein Datenformat, das den Aufbau einer Textdatei zur Speicherung oder zum Austausch einfach strukturierter Daten beschreibt. (engl. comma-seperated values)
\newline (Quelle: de.wikipedia.org 11.11.2015)}
}
\newglossaryentry{GUI}
{
name=GUI,
description={Ist eine Grafische Benutzeroberfläche. (engl. Graphical User Interface)}
}
\newglossaryentry{HTML}
{
name=HTML,
description={Ist eine Textbasierte Auszeichnungssprache zur Strukturierung digitaler Dokumente. (engl. Hypertext Markup Language)
\newline (Quelle: de.wikipedia.org 11.11.2015)}
}
\newglossaryentry{IEEE}
{
name=IEEE,
description={Ist ein weltweiter Berufsverband, der sich um die Förderung technologischer Innovationen zum Nutzen der Menschheit kümmert. (engl. Insitute of Electrical and Electronics Engineers)
\newline (Quelle: de.wikipedia.org 11.11.2015)}
}
\newglossaryentry{Java}{
  name=Java,
  description={Ist eine Programmiersprache.}
}

\newglossaryentry{Javascript}{
  name=Javascript,
  description={Ist eine Sprache, mit der dynamisches \gls{HTML} umgesetzt werden kann.
  \newline (Quelle: de.wikipedia.org 11.11.2015)}
}

\newglossaryentry{JQuery}{
  name=JQuery,
  description={Ist eine freie \gls{Javascript} Bibliothek.
  \newline (Quelle: de.wikipedia.org 11.11.2015)}
}
\newglossaryentry{Open Source}{
	name=Open Source,
	description={Wird als Begriff für Software verwendet, deren Programmcode für jeden lizenzfrei zugänglich ist.
	\newline (Quelle: de.wikipedia.org 11.11.2015)}
}

\newglossaryentry{Paket}{
  name=Paket,
  description={Ist ein Ordner, der Trainingsdatensatz, Testdatensatz, Modell und Ergebnisdatensätze enthält.}
}

\newglossaryentry{REST}{
  name=REST,
  description={Ist ein Programmierstil für Webanwendungen. (engl. Represational State Transfer)
  \newline (Quelle: de.wikipedia.org 11.11.2015)}
}

\newglossaryentry{RDF}{
  name=RDF,
  description={Dies bezeichnet die technische Herangehensweise, um logische Aussagen zu treffen. (engl. Resource Description Framework)
  \newline (Quelle: de.wikipedia.org 11.11.2015)}
}

\newglossaryentry{Salt}{
  name=Salt,
  description={Ist eine zufällige gewählte Zeichenfolge, die zu einer Verschlüsselung zusätzliche Sicherheit bringt.
  \newline (Quelle: de.wikipedia.org 11.11.2015)}
}

\newglossaryentry{SHA-256}{
  name=SHA-256,
  description={Ist ein sicherer 256-Bit-Algorithmus zum verschlüsseln. (engl. Secure hash algorithm)
   \newline (Quelle: de.wikipedia.org 11.11.2015)}
}

\newglossaryentry{SSL}{
  name=SSL,
  description={Ist ein Verschlüsselungsprotokoll zur sicheren Datenübertragung. (engl. Secure Sockets Layer) 
   \newline (Quelle: de.wikipedia.org 11.11.2015)
  }
}

\newglossaryentry{Superadmin}{
  name=Superadmin,
  description={Ist die Person, die das gesamte System überwacht und wartet. Diese Person besitzt alle Rechte des Systems.}
}

\newglossaryentry{Webbrowser}{
  name=Webbrowser,
  description={Ist ein Programm zur Darstellung von Webseiten.}
}

\newglossaryentry{WEKA}{
  name=WEKA,
  description={Ist ein Softwaretool, das verschiedene Techniken aus den Bereichen Maschinelles Lernen und Data-Mining bereitstellt. (engl. Waikato Environment for Knowledge Analysis)
  \newline(Quelle: de.wikipedia.org 11.11.2015) }
}

