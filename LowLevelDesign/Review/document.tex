\documentclass{article}
\usepackage[T1]{fontenc}
\usepackage[utf8]{inputenc}
\title{Review Document -- Requirements Document}
\author{David Klopp \and Christian Stricker \and Markus Vieth}
\date{\today}
%Requirement, Problem, Solution
\newcommand{\RPS}[3]{\subsection{#1} \subsubsection*{Problem:} #2 \subsection*{Lösung:} #3} 

\begin{document}
\maketitle
\cleardoublepage
\tableofcontents 
\newpage
\section{Introduction}
Der Gruppe 11 sind einige Fehler beim Formulieren des Low-Level Documents unterlaufen, welche im folgenden behandelt werde.
\section{User-Requirements}

\RPS{Klasse Object}{%
Klassenname "Object" ist irreführend. Es könnte die Java Klasse Object gemeint sein, von der abgeleitet werden soll. %
}{%
Ändere den Klassennamen in "SystemObject" umd anzuzeugen, das eine eigene implementierte Klasse gemeint ist.%
}

\RPS{Sicherheit}{%
	ClientCrypter und ServerCrypter sind unnötig, da eine Https-Verbindung zwischen Client und Server und zwischen verschiedenen Servern aufgebaut wird. Die Crypter würden außerdem innerhalb vom Server zwischen den Client und Server verschlüsseln, die Performance sinkt. %
}{%
ClientCrypter und ServerCrypter entfernen.%
}

\RPS{Sequenzdiagramm}{%
	Die Funktion "getModel(id)" gehört zu dem PackageManager und nicht an das ModelInterface, es kann nicht das Interface aufrufen. Der PackageManager Existiert schon, bevor er vom ModelInterface aufgerufen wird. %
}{%
Funktion "getModel(id)" in den PackageManager packen und Startzeitpunkt des PackageManager nach hinten setzen.%
}

\RPS{Weka-Anbindung}{%
	Eine Weka-Anbindung vom System fehlt komplett %
}{%
Weka-Library downloaden und eine mit einer Schnittstelle in das System integrieren%
}

\RPS{Klasse Object}{%
	Problem %
}{%
Lösung%
}

\RPS{Klasse Object}{%
	Problem %
}{%
Lösung%
}

\RPS{Klasse Object}{%
	Problem %
}{%
Lösung%
}

\RPS{Klasse Object}{%
	Problem %
}{%
Lösung%
}


\end{document}

%%% Local Variables:
%%% mode: latex
%%% TeX-master: t
%%% End:
