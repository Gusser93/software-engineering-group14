%UR
\newcounter{urcounter} %Erzeugt neuen Counter für URs
\newcommand{\UR}[2]{
	\ifthenelse
	{\value{urcounter}<9}		%if counter kleiner 9
	{\paragraph{UR00\theurcounter}} %dann zwei Nullen drann hängen
	{\paragraph{UR0\theurcounter}}  %sonst eine Null, der Fall keine Null wird ignoriert, da er nicht auftreten wird
\addtocounter{urcounter}{1}%Erhöht den Zähler um 1
\begin{description}
	\item[Statement] \textit{#1}
	\item[Priority] \textit{#2}
\end{description}}		

%NFR
\newcounter{nfrcounter} %Erzeugt neuen Counter für URs
\newcommand{\NFR}[2]{
	\ifthenelse
	{\value{nfrcounter}<9}		%if counter kleiner 9
	{\paragraph{NFR00\thenfrcounter}} %dann zwei Nullen drann hängen
	{\paragraph{NFR0\thenfrcounter}}  %sonst eine Null, der Fall keine Null wird ignoriert, da er nicht auftreten wird
	\addtocounter{nfrcounter}{1}		%Erhöht den Zähler um 1
	\begin{description}
		\item[Statement] \textit{#1}
		\item[Priority] \textit{#2}
	\end{description}}		

%FR
\newcounter{frcounter} %Erzeugt neuen Counter für URs
\newcommand{\FR}[3]{
	\ifthenelse
	{\value{frcounter}<9}		%if counter kleiner 9
	{\paragraph{FR00\thefrcounter}} %dann zwei Nullen drann hängen
	{\paragraph{FR0\thefrcounter}}  %sonst eine Null, der Fall keine Null wird ignoriert, da er nicht auftreten wird
	\addtocounter{frcounter}{1}		%Erhöht den Zähler um 1
	\begin{description}
		\item[Statement] \textit{#1 (siehe user requirement UR#2)}
		\item[Priority] \textit{#3}
	\end{description}}		
	
	\DeclareDocumentCommand{\newdualentry}{ O{} O{} m m m m } {
  \newglossaryentry{gls-#3}{name={#5},text={#5\glsadd{#3}},
    description={#6},#1
  }
  \newacronym[see={[Glossary:]{gls-#3}},#2]{#3}{#4}{#5\glsadd{gls-#3}}
}