\newglossaryentry{session}
{
	name=session,
	description={A session consists of everything which belongs to one concrete conference}
}
\newdualentry{GUI}{GUI}{Graphical User Interface}{folgt}
\newglossaryentry{Plugin}
{
	name=Plugin,
	description={Ein kleines Softwareprogramm, das in eine größere Anwendung integriert werden kann. (Quelle: Duden.de: 11.11.2015)},
	plural=Plugins
}
\newglossaryentry{Plugins}
{
	name=Plugin,
	description={A plugin is an extension to any sort of software that adds additional features or alters existing ones},
	plural=Plugins
}
\newglossaryentry{Flexdock}
{
	name=Plugin,
	description={A plugin is an extension to any sort of software that adds additional features or alters existing ones},
	plural=Plugins
}
\newdualentry{HTTPS}{HTTPS}{HyperText Transfer Protocol Secure}{HyperText Transfer Protocol Secure (englisch für sicheres Hypertext-Übertragungsprotokoll) ist ein Kommunikationsprotokoll im World Wide Web, um Daten abhörsicher zu übertragen. Unter Verwendung des SSL-Handshake-Protokolls findet zunächst eine geschützte Identifikation und Authentifizierung der Kommunikationspartner statt. Anschließend wird mit Hilfe asymmetrischer Verschlüsselung oder des Diffie-Hellman-Schlüsselaustauschs ein gemeinsamer symmetrischer Sitzungsschlüssel ausgetauscht. Dieser wird schließlich zur Verschlüsselung der Nutzdaten verwendet. (Quelle: de.wikipedia.org: 11.11.2015)}

\newdualentry{HTML5}{HTML5}{Hyper Text Markup Language 5}{HTML5 ist die fünfte Fassung der Hypertext Markup Language (engl. für Hypertext-Auszeichnungssprache), einer Computersprache zur Auszeichnung und Vernetzung von Texten und anderen Inhalten elektronischer Dokumente, vorwiegend im World Wide Web. (Quelle: de.wikipedia.org: 11.11.2015)}

\newglossaryentry{JavaScript}
{
name=JavaScript,
description={JavaScript ist eine Skriptsprache, die ursprünglich für dynamisches HTML in Webbrowsern entwickelt wurde, um Benutzerinteraktionen auszuwerten, Inhalte zu verändern, nachzuladen oder zu generieren und so die Möglichkeiten von HTML und CSS zu erweitern. (Quelle: de.wikipedia.org: 11.11.2015)}
}

\newdualentry{API}{API}{Application programming interface}{The application programming interface describes how to interact with a system. The interface provides methodes which can be accessed from outside the system}

\newdualentry{ICE}{ICE}{Interactive Connectivity Establishment}{Interactive Connectivity Establishment (ICE) is a technique to establish connections with clients behind a router or firewall}

\newglossaryentry{recordable plugins}{
	name=recordable plugin,
	description={These are plugins which can be recorded. In case of start the recording of such a plugin the plugin starts to record himself from this moment on in a seperatly file on the server}
}

\newglossaryentry{Web-Applikation}{name={Web-Applikation},description={folgt}}

\newdualentry{UI}{UI}{User-Interface}{folgt}

\newdualentry{REST}{REST}{Representational State Transfer}{folgt}

\newdualentry{RDF}{RDF}{Resource Description Framework}{folgt}

\newglossaryentry{Gast}{name={Gast},description={Bin mir unsicher, ob wir das brauchen.}, plural=Gäste}

%\newglossaryentry{Nutzer}{name={Nutzer},description={folgt},plural=Nutzer}
\newglossaryentry{admin}{name={Administrator},description={folgt},plural=Administratoren}

\newglossaryentry{Model}{name={Modell},description={folgt},plural=Modelle}


\newglossaryentry{Browser}{name={Browser},description={folgt},plural=Browser}

\newglossaryentry{Datensatz}{name={Datensatz},description={folgt},plural=Datensätze}

\newglossaryentry{Paket}{name={Paket},description={folgt},plural=Pakete}

\newglossaryentry{client}{name={Client},description={folgt},plural=Clients}

\newglossaryentry{clustering}{name={Clustering},description={folgt},plural=Clusterverfahren}

\newglossaryentry{server}{name=Server, description={folgt}, plural=Server}

\newglossaryentry{backup}{name=Backup, description={folgt}, plural=Backups}

\newglossaryentry{downtime}{name=Downtime, description={folgt}, plural=Downtimes}

\newdualentry{ssl}{SSL}{Secure Sockets Layer}{folgt}

\newdualentry{IEEE}{IEEE}{Institute of Electrical and Electronics Engineers}{folgt}

\newglossaryentry{openssl}{name=OpenSSL, description={folgt}, plural=OpenSSL}

\newglossaryentry{opensource}{name={Open Source}, description={folgt}}

\newglossaryentry{Java}{name=Java, description={folgt}, plural=Java}

\newglossaryentry{Interface}{name=Interface, description={folgt}, plural=Interfaces}

\newdualentry{URI}{URI}{Uniform Resource Identifier}{folgt}
