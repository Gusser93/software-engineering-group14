\newglossaryentry{session}
{
	name=session,
	description={A session consists of everything which belongs to one concrete conference}
}
\newdualentry{GUI}{GUI}{Graphical User Interface}{Grafische Benutzeroberfläche oder auch grafische Benutzerschnittstelle (von englisch graphical user interface) bezeichnet eine Form von Benutzerschnittstelle eines Computers. Sie hat die Aufgabe, Anwendungssoftware auf einem Rechner mittels grafischer Symbole, Steuerelemente oder auch Widgets genannt, bedienbar zu machen. }
\newglossaryentry{Plugin}
{
	name=Plugin,
	description={Ein kleines Softwareprogramm, das in eine größere Anwendung integriert werden kann. (Quelle: Duden.de: 11.11.2015)},
	plural=Plugins
}
\newdualentry{HTTPS}{HTTPS}{HyperText Transfer Protocol Secure}{HyperText Transfer Protocol Secure (englisch für sicheres Hypertext-Übertragungsprotokoll) ist ein Kommunikationsprotokoll im World Wide Web, um Daten abhörsicher zu übertragen. Unter Verwendung des SSL-Handshake-Protokolls findet zunächst eine geschützte Identifikation und Authentifizierung der Kommunikationspartner statt. Anschließend wird mit Hilfe asymmetrischer Verschlüsselung oder des Diffie-Hellman-Schlüsselaustauschs ein gemeinsamer symmetrischer Sitzungsschlüssel ausgetauscht. Dieser wird schließlich zur Verschlüsselung der Nutzdaten verwendet. (Quelle: de.wikipedia.org: 11.11.2015)}

\newdualentry{HTML5}{HTML5}{Hyper Text Markup Language 5}{HTML5 ist die fünfte Fassung der Hypertext Markup Language (engl. für Hypertext-Auszeichnungssprache), einer Computersprache zur Auszeichnung und Vernetzung von Texten und anderen Inhalten elektronischer Dokumente, vorwiegend im World Wide Web. (Quelle: de.wikipedia.org: 11.11.2015)}

\newglossaryentry{JavaScript}
{
name=JavaScript,
description={JavaScript ist eine Skriptsprache, die ursprünglich für dynamisches HTML in Webbrowsern entwickelt wurde, um Benutzerinteraktionen auszuwerten, Inhalte zu verändern, nachzuladen oder zu generieren und so die Möglichkeiten von HTML und CSS zu erweitern. (Quelle: de.wikipedia.org: 11.11.2015)}
}

\newdualentry{API}{API}{Application programming interface (Anwendungsprogrammierschnittstelle)}{Eine Programmierschnittstelle, genauer Schnittstelle zur Anwendungsprogrammierung, ist ein Programmteil, der von einem Softwaresystem anderen Programmen zur Anbindung an das System zur Verfügung gestellt wird.}

\newglossaryentry{Web-Applikation}{name={Web-Applikation},description={Eine Webanwendung (auch Webapplikation oder kurz Web-App) ist ein Anwendungsprogramm, das in einem Webbrowser angezeigt und bedient wird. Webanwendungen liegen auf einem Webserver, auf den z. B. über das Internet oder ein Intranet zugegriffen werden kann, und werden von dort vom Klienten geladen.}}

\newdualentry{UI}{UI}{User-Interface}{Eine computergestützte Benutzerschnittstelle oder Benutzeroberfläche, eigentlich Benutzungsschnittstelle, ist der Teil eines Computerprogramms, der mit dem Benutzer kommuniziert. In DIN EN ISO 9241-110 ist der Begriff der Benutzungsschnittstelle definiert als „alle Bestandteile eines interaktiven Systems (Software oder Hardware), die Informationen und Steuerelemente zur Verfügung stellen, die für den Benutzer notwendig sind, um eine bestimmte Arbeitsaufgabe mit dem interaktiven System zu erledigen.“}

\newdualentry{REST}{REST}{Representational State Transfer}{Ein Programmierparadigma für verteilte Systeme, insbesondere für Webservices. REST ist eine Abstraktion der Struktur und des Verhaltens des World Wide Web. REST fordert, dass eine URI (Adresse) genau einen Seiteninhalt repräsentiert, und dass ein Web-/REST-Server auf mehrfache Anfragen mit demselben URI auch mit demselben Webseiteninhalt antwortet.}

\newdualentry{RDF}{RDF}{Resource Description Framework}{Das Resource Description Framework (engl. sinngemäß „System zur Beschreibung von Ressourcen“) bezeichnet eine technische Herangehensweise im Internet zur Formulierung logischer Aussagen über beliebige Dinge (Ressourcen)}

\newglossaryentry{Gast}{name={Gast},description={Anonymer Nutzer mit eingeschränkten Rechten.}, plural=Gäste}

\newglossaryentry{Nutzer}{name={Nutzer},description={Person welche das System nutzt},plural=Nutzer}
\newglossaryentry{admin}{name={Administrator},description={Ein Administrator (kurz Admin) ist ein Nutzer mit speziellen Aufgaben, hauptsächlich das Verwalten und, je nach Art und Weise, teilweise Redigieren von Inhalten. Um diese Aufgaben zu erfüllen und die jeweiligen Benutzungsrichtlinien durchzusetzen, hat ein Administrator erweiterte Benutzerrechte.},plural=Administratoren}

\newglossaryentry{Model}{name={Modell},description={Algorithmus, welcher durch Datensätze angelernt wurde und Vorhersagen über neue Datensätze treffen kann},plural=Modelle}


\newglossaryentry{Browser}{name={Browser},description={Webbrowser oder allgemein auch Browser (aus engl. to browse, ‚stöbern, schmökern, umsehen‘, auch ‚abgrasen‘) sind spezielle Computerprogramme zur Darstellung von Webseiten im World Wide Web oder allgemein von Dokumenten und Daten.},plural=Browser}

\newglossaryentry{Datensatz}{name={Datensatz},description={Ein Datensatz ist (beispielsweise nach Mertens) eine Gruppe von inhaltlich zusammenhängenden (zu einem Objekt gehörenden) Datenfeldern, z. B. Artikelnummer und Artikelname},plural=Datensätze}

\newglossaryentry{Paket}{name={Paket},description={Durch einen Nutzer generierte Zusammenfassung verschiedener Datensätze, Modelle, Vorhersagen und weiteren relevanten Daten},plural=Pakete}

\newdualentry{SPARQL}{SPARQL}{SPARQL Protocol And RDF Query Language}{SPARQL ist eine graphenbasierte Abfragesprache für \gls{RDF}}

\newglossaryentry{client}{name={Client},description={Ein Client (deutsch „Kunde“, auch clientseitige Anwendung oder Clientanwendung) bezeichnet ein Computerprogramm, das auf dem Endgerät eines Netzwerks ausgeführt wird und mit einem Zentralrechner Server kommuniziert. Man nennt auch ein Endgerät selbst, das Dienste von einem Server abruft, Client.},plural=Clients}

\newglossaryentry{clustering}{name={Clustering},description={folgt},plural=Clusterverfahren}

\newglossaryentry{server}{name=Server, description={Server wird ein in ein Rechnernetz eingebundenes Rechnersystem mit zugehörigem Betriebssystem bezeichnet, das Clients bedient oder Server (Software) beherbergt.}, plural=Server}

\newglossaryentry{backup}{name=Backup, description={Datensicherung (englisch backup) bezeichnet das Kopieren von Daten in der Absicht, diese im Fall eines Datenverlustes zurückkopieren zu können.}, plural=Backups}

\newglossaryentry{downtime}{name=Downtime, description={Downtime (engl. für Stillstandszeit, Ausfallzeit, Abstellzeit) ist die gebräuchliche Bezeichnung der Zeit, in der ein System, insb. ein Computersystem, nicht verfügbar bzw. nicht funktionstüchtig ist. Man unterscheidet zwischen geplanter und ungeplanter Downtime}, plural=Downtimes}

\newdualentry{ssl}{SSL}{Secure Sockets Layer}{Transport Layer Security (TLS, deutsch Transportschichtsicherheit), weitläufiger bekannt unter der Vorgängerbezeichnung Secure Sockets Layer (SSL), ist ein hybrides Verschlüsselungsprotokoll zur sicheren Datenübertragung im Internet. Seit Version 3.0 wird das SSL-Protokoll unter dem neuen Namen TLS weiterentwickelt und standardisiert}

\newdualentry{IEEE}{IEEE}{Institute of Electrical and Electronics Engineers}{Das Institute of Electrical and Electronics Engineers (IEEE, meist als „i triple e“ gesprochen) ist ein weltweiter Berufsverband von Ingenieuren hauptsächlich aus den Bereichen Elektrotechnik und Informationstechnik mit juristischem Sitz in New York City und Betriebszentrale in Piscataway, New Jersey. Er ist Veranstalter von Fachtagungen, Herausgeber diverser Fachzeitschriften und bildet Gremien für die Standardisierung von Techniken, Hardware und Software.}

\newglossaryentry{openssl}{name=OpenSSL, description={OpenSSL, ursprünglich SSLeay, ist eine freie Software für Transport Layer Security, ursprünglich Secure Sockets Layer (SSL).}, plural=OpenSSL}

\newglossaryentry{opensource}{name={Open Source}, description={Open Source bzw. quelloffen wird als Begriff für Software verwendet, deren Quelltext offenliegt und frei verfügbar ist. Im engeren Sinne steht Open Source Software (OSS) für Software die die Definition der Open Source Initiative (OSI) erfüllt, beispielsweise darüber das diese Software einer OSI anerkannten Open-Source-Softwarelizenz unterliegt.}}

\newglossaryentry{Java}{name=Java, description={Java ist eine objektorientierte Programmiersprache und eine eingetragene Marke des Unternehmens Sun Microsystems}, plural=Java}

\newglossaryentry{Interface}{name=Interface, description={Die Schnittstelle oder das Interface ( englisch für Grenzfläche) ist der Teil eines Systems, welcher der Kommunikation dient.}, plural=Interfaces}

\newglossaryentry{Software}{name=Software, description={Software (dt. = weiche Ware [von] soft = leicht veränderbare Komponenten [...], Komplement zu ‚Hardware‘ für die physischen Komponenten)[1] ist ein Sammelbegriff für Programme und die zugehörigen Daten}, plural=Software}

\newglossaryentry{dropdown}{name=dropdown-Menü, description={Ein Dropout-Menü, auch Dropdown-, Pulldown- oder Aufklapp-Menü, ist ein Steuerelement einer grafischen Benutzeroberfläche. Dabei wird über einen Mausklick auf eine Schaltfläche einer Menüleiste oder Symbolleiste ein Untermenü angezeigt.}, plural=dropdown Menüs}

\newglossaryentry{Gruppe}{name=Gruppe, description={Eine von Nutzern erstellte Gruppierung verschiedener Nutzer}, plural=Gruppe}

\newglossaryentry{domain}{name=Domain, description={Eine Domain (von englisch domain ‚Bereich‘, ‚Domäne‘) ist ein zusammenhängender Teilbereich des hierarchischen Domain Name System (DNS).}, plural=Domains}

\newdualentry{URI}{URI}{Uniform Resource Identifier}{Ein Uniform Resource Identifier (Abk. URI, englisch für einheitlicher Bezeichner für Ressourcen) ist ein Identifikator und besteht aus einer Zeichenfolge, die zur Identifizierung einer abstrakten oder physischen Ressource dient. URIs werden zur Bezeichnung von Ressourcen (wie Webseiten, sonstigen Dateien, Aufruf von Webservices, aber auch z. B. E-Mail-Empfängern) im Internet und dort vor allem im WWW eingesetzt. Der aktuelle Stand ist als RFC 3986 publiziert.}
