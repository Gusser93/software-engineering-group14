\newglossaryentry{session}
{
	name=session,
	description={A session consists of everything which belongs to one concrete conference}
}
\newglossaryentry{GUI}
{	
	name=GUI,
	description={Graphical User Interface}
}
\newglossaryentry{Plugins}
{
	name=Plugins,
	description={A plugin is an extension to any sort of software that adds additional features or alters existing ones}
}
\newglossaryentry{garbageCollection}
{
name=garbage collection,
description={Garbage collection is a routine to search unused files and data sets. Found files will be deleted}
}
\newglossaryentry{HTTPS}
{
name=Hypertext Transfer Protocoll Secure,
description={Hypertext Transfer Protocol Secure (HTTPS) is a communications protocol for secure communication over a computer network, with especially wide deployment on the Internet. Technically, it is not a protocol in and of itself; rather, it is the result of simply layering the Hypertext Transfer Protocol (HTTP) on top of the SSL/TLS protocol, thus adding the security capabilities of SSL/TLS to standard HTTP communications. (source: en.wikipedia.org: 11.11.2013)}
}
\newglossaryentry{HTML5}
{
name=Hyper Text Markup Language 5,
description={HTML 5 is a markup language which is used for structuring and presenting content for the World Wide Web}
}
\newglossaryentry{JavaScript}
{
name=JavaScript,
description={JavaScript is an interpreted computer programming language. As part of web browsers, implementations allow client-side scripts to interact with the user, control the browser, communicate asynchronously, and alter the document content that is displayed. It has also become common in server-side programming, game development, and the creation of desktop applications. (source: en.wikipedia.org : 11.11.2013)}
}
\newglossaryentry{webRTC}
{
name=webRTC,
description={WebRTC is a free, open project that enables web browsers with real-time communications (RTC) capabilities via simple Javascript APIs. The WebRTC components have been optimized to best serve this purpose. WebRTC also uses HTML 5. (source: http://www.webrtc.org/: 11.11.2013)}
}
\newglossaryentry{API}{
  name=Application programming interface,
  description={The application programming interface describes how to interact with a system. The interface provides methodes which can be accessed from outside the system}
}

\newglossaryentry{Flexdock}{
  name=Flexdock,
  description={Flexdock is the name of a framework where subwindows can be dragged freely}
}

\newglossaryentry{ICE}{
  name=Interactive Connectivity Establishment,
  description={Interactive Connectivity Establishment (ICE) is a technique to establish connections with clients behind a router or firewall}
}
\newglossaryentry{recordable plugins}{
	name=recordable plugin,
	description={These are plugins which can be recorded. In case of start the recording of such a plugin the plugin starts to record himself from this moment on in a seperatly file on the server}
}
\newglossaryentry{Web-Applikation}{name={Web-Applikation},description={folgt}}
\newacronym{UI}{UI}{User-Interface}
\newacronym{REST}{REST}{Representational State Transfer}
\newacronym{RDF}{RDF}{Resource Description Framework}
\newglossaryentry{Gast}{name={Gast},description={Bin mir unsicher, ob wir das brauchen.}}
%\newglossaryentry{Nutzer}{name={Nutzer},description={folgt},plural=Nutzer}
\newglossaryentry{admin}{name={Administrator},description={folgt},plural=Administratoren}
\newglossaryentry{Model}{name={Model},description={folgt},plural=Modelle}
\newglossaryentry{Backup}{name={Backup},description={folgt},plural=Backups}
\newglossaryentry{Browser}{name={Browser},description={folgt},plural=Browser}
\newglossaryentry{Datensatz}{name={Datensatz},description={folgt},plural=Datensätze}
\newglossaryentry{Paket}{name={Paket},description={folgt},plural=Pakete}