%1
\FR{Das System kann über \gls{REST} und \gls{HTTPS} bedient werden.}{001}{A}

\FR{Das System soll die folgenden Desktop Browser unterstützen: Firefox 38, Google Chrome C42, Microsoft Internet Explorer 8, Microsoft Edge 12.}{001}{A}


%2
\FR{Bei gleichen Eingaben sollen gleiche Ergebnisse bzw. Vorhersagen generiert werden.}{002}{A}

\FR{Die Funktionsfähigkeit der nicht standardgemäß implementierten Algorithmen soll durch den \gls{admin} verifiziert werden.}{002}{A}


%3
\FR{Das System soll das Hochladen von Daten ermöglichen, sowie das Festsetzen von Berechtigungen für Selbige.}{003}{A}

\FR{Datensätze sollen hierbei entweder als csv oder raff Datei hochgeladen werden können.}{003}{A}

\FR{Sollten zu viele Uploads gleichzeitig gestartet werden, soll das System diese in eine Warteschlange einfügen.}{003}{A}

\FR{Die hochgeladenen Daten sollen validiert werden.}{003}{B}

\FR{Das System generiert einen eindeutige \gls{URI} für jede Datei, über die der \gls{Nutzer} auf jene zugreifen kann.}{003}{A}


%4
\FR{Die Datensätze sowie die Auswahl des Algorithmus erfolgt über eine tabellarisch angeordnete Eingabemaske.}{004}{A}

\FR{Die eingegebenen Daten müssen nicht vom System auf ihre Richtigkeit geprüft werden.}{004}{A}


%5
\FR{Das System soll plattformübergreifend einsetzbar sein.}{005}{A}


%6
\FR{Das System soll Java \glspl{Interface} zur Erstellung von \glspl{Plugin} zur Verfügung stellen. Jedes Interface soll mindestens festlegen welche Datensätze das \gls{Plugin} verarbeiten kann und welche Rückgabewerte zu erwarten sind.}{006}{A}


%7
\FR{Das System soll eine Ansicht zur Benutzerverwaltung beinhalten, über die personenbezogene Daten und Einstellungen geändert werden können. Im Falle des \gls{admin} sollen weitere Konfigurationsoptionen für z.B. \glspl{Gast} zur Verfügung gestellt werden.}{007}{A}


%8
\FR{Das System soll eine Ansicht zur Verwaltung der vom \gls{Nutzer} hochgeladenen Daten beinhalten. Im Falle des \gls{admin} sollen sämtliche Dateien auf dem \gls{server} verwaltet werden können.}{008}{A}


%9
\FR{Das System soll ein Standard \gls{Plugin} für die grafische Ausgabe von Algorithmen beinhalten.}{009}{A}

\FR{Weitere grafische \glspl{Plugin} sollen nach der Überprüfung durch den \gls{admin} in das System integriert werden können.}{009}{A}


%10
\FR{Das System muss \gls{REST} konform sein, um die serverübergreifende Kommunikation zu gewährleisten.}{0010}{A}


%11
\FR{Das System soll die Registrierung von Nutzern über eine gültige E-Mail-Adresse, Benutzername und Passwort ermöglichen.}{011}{A}

\FR{Bei der Registrierung neuer \gls{Nutzer} soll vorher geprüft werden ob bereits ein Benutzer mit dem angegebenen Name oder der gleichen E-Mail Adresse existiert. Sollte dies zutreffen soll eine entsprechende Fehlermeldung ausgegeben werden.}{011}{A}


%12
\FR{Das System soll einen Gastnutzer bereitstellen, der gleichzeitig mehrfach verwendet werden kann.}{012}{A}

\FR{Der \gls{Gast} darf keinen Zugriff auf den Menüpunkte zur Verwaltung der Gruppen haben.}{012}{A}


%13
\FR{Der \gls{admin} soll über das Einstellungsmenü die Möglichkeit haben die Größe des Datenuploads für \glspl{Gast} über ein Eingabefeld zu beschränken.}{013}{A}

\FR{Der \gls{admin} soll die Laufzeit von Algorithmen die der \gls{Gast} ausführt, über das Einstellungsmenü, auf eine definierte Höchstlaufzeit festlegen können. Die Laufzeit muss mindestens 10 Minuten betragen.}{013}{A}


%14
\FR{Der Nutzer soll über das Dateiverwaltungsmenü die Möglichkeit haben neue Pakete zu erstellen. Beim Erstellen des Pakets  soll er die Dateien, die jenes Paket beinhalten soll, auswählen können.}{014}{A}

\FR{Der Nutzer soll in einem Untermenü "Pakete", der Dateiverwaltung, alle öffentlich zugänglichen, sowie seine eigenen Pakete einsehen und herunterladen können.}{014}{A}

\FR{Die Pakete sollen als zip Dateien heruntergeladen werden können.}{014}{A}

\FR{Datensätze sollen als csv oder raff Datei heruntergeladen werden können.}{014}{A}


%15
\FR{Die Daten sollen in Tripeln abgespeichert werden.}{015}{A}

\FR{Die Datenbank soll lediglich Verweise auf Jar Dateien für die Algorithmen beinhalten.}{015}{A}


%16 
\FR{Mehrere Simulationen sollen gleichzeitig ausgeführt werden können.}{016}{A}

\FR{Der \gls{admin} soll die Anzahl an Simulationen über das Einstellungsmenü festlegen können.}{016}{A}


%17
\FR{Monatlich ist das System planmäßig für 10 Minuten nicht erreichbar um Backups zu erstellen.}{017}{A}

\FR{Im Falle eines nicht planmäßigen Ausfalls des Systems soll der \gls{admin} über seine E-Mail-Adresse benachrichtigt werden.}{017}{A}

\FR{Das System soll plattformunabhängig und nicht an spezifische Hardware gebunden sein, um eine schnelle Portierung der Software auf neue \gls{server} zu gewährleisten.}{017}{A}


%18


%19
\FR{Die Algorithmen sollen in Form einer Jar Datei in das System integriert werden.}{019}{A}

\FR{Über die \gls{Plugin}-\gls{API} soll jedem Algorithmus eine Priorität zugewiesen werden können, um diesem mehr Rechnerkapazität zur Verfügung zu stellen.}{019}{B}


%20
\FR{Zur besseren globalen Verständigung soll die Systemsprache standardmäßig auf Englisch sein.}{020}{A}

\FR{Die \gls{domain} des \gls{server} muss weltweit erreichbar sein }{020}{A}

%21
\FR{Das System soll über iOS ab Version 7.0, sowie über Android ab Version 4.0 über den integrierten Browser des jeweiligen Betriebssystems erreichbar sein.}{021}{A}

\FR{Das System soll über eine mobile Website verfügen.}{021}{B}


%22
\FR{Die einzigen Vorraussetzungen zur Benutzung des Webservices soll eine bestehende Internetverbindung, sowie ein aktueller Browser sein.}{022}{A}

\FR{Das System soll ohne spezielle Software auf dem Rechner des \gls{client} auskommen.}{022}{A}

\FR{Sämtliche Daten werden standardgemäß auf dem \gls{server}, nicht auf dem \gls{client} gespeichert.}{022}{A}


%23
\FR{Der \gls{admin} soll über das Einstellungsmenü Nutzergruppen priorisieren und privilegieren können.}{023}{A}

\FR{Alle Daten des \gls{Gast} sind öffentlich zugänglich.}{023}{A}

\FR{Der \gls{Nutzer} soll in der Lage sein Nutzergruppen, denen er angehört, selbst zu verwalten.}{023}{A}


%24
\FR{Das System soll dem \gls{admin} eine Übersicht über alle laufenden Algorithmen bereitstellen.}{024}{A}

\FR{Algorithmen können sowohl vom \gls{admin}, wie auch vom \gls{Nutzer}, der diese gestartet hat, abgebrochen werden.}{024}{A}


%25
\FR{Die unterstützten Dateiformate eines Modells sollen über eine \gls{Java}-\gls{API} abgefragt werden können.}{025}{A}

\FR{Die \gls{Plugin}-\gls{API} für grafische Erweiterungen soll eine Schnittstelle zum Vergleich zweier Modelle bereitstellen.}{025}{A}


%26
\FR{Der Java-Quellcode soll sich an die "Google Java Style Guidelines" halten.}{026}{A}

\FR{Der HTML- und CSS-Quellcode soll sich an die Standards des  World Wide Web Consortium (W3C) halten.}{026}{A}

\FR{Der Quellcode soll opensource zur Verfügung gestellt werden.}{026}{A}


%27
\FR{Das System soll ein Hilfemenü mit FAQ beinhalten.}{027}{A}

\FR{Der Software soll ein elektronisches Handbuch zur Verwendung der Selbigen beiliegen.}{027}{A}

\FR{Nach der Registrierung eines \gls{Nutzer}s soll dieser durch ein kurzes Tutorial in die Bedienung der Software eingeführt werden.}{027}{B}

%28
\FR{Die Lizenzinformationen eines Models sollen zusammen mit diesem in der Datenbank abgespeichert werden.}{028}{A}

\FR{Die Lizenzinformationen zu einem Modell sollen von jedem \gls{Nutzer} eingesehen werden können.}{028}{A}

\FR{Die Benutzung des Modells wird durch die angegebenen Lizenzinformationen auf bestimmte Nutzergruppen limitiert.}{028}{A}

\FR{Nach der Generierung eines Modells soll der Nutzer automatisch ein \gls{dropdown}-Menü zum Hinzufügen der Lizenzinformationen angezeigt bekommen.}{028}{A}


%29
\FR{Das System, sowie der zugehörige Quellcode, sollen für jeden frei zugänglich sein.}{029}{A}

\FR{Ein kommerzieller Vertrieb der Software ist untersagt.}{029}{A}


%30
\FR{Sämtlicher Dateiaustausch zwischen \gls{Nutzer} und \gls{server} erfolgen über eine sichere HTTPS Verbindung die die Daten mittels SSL verschlüsselt.}{030}{A}


%31
\FR{Die Algorithmen sollen als Jar-Datei heruntergeladen werden können.}{031}{A}


%32
\FR{Beim Upload eines Algorithmus soll der \gls{admin} automatisch via E-Mail über diesen Umstand informiert werden.}{032}{A}

\FR{Zu jedem Algorithmus wird der Uploader hinterlegt, für den Fall das der \gls{admin} diesen \gls{Nutzer} kontaktieren muss.}{032}{A}

%33
% Ausreichend behandelt in obigen Punkten 

%34
\FR{In regelmäßigen Abständen soll das System den Status des letzten Backups mit dem entsprechenden Sicherungsserver abgleichen.}{034}{A}

\FR{Planmäßig soll sich das System automatisch einmal im Monat sichern. Sollte bei diesem Prozess ein Fehler auftreten, soll der \gls{admin} via E-Mail darüber informiert werden.}{034}{A}

%35
\FR{Das System soll variabel genug programmiert sein um mögliche weitere Systemsprachen hinzuzufügen.}{035}{A}

%36
\FR{Jeder \gls{Nutzer} soll über genau eine eindeutige URL auf eine gewisse Ressource zugreifen können.}{036}{A}
