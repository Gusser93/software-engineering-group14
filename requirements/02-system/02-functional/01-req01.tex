\FR{The system makes use of \gls{webRTC} to provide the video, voice and text communication.
	(see user requirement UR001)}{A}

\FR{When having a conference and one user clicks on another users name, a context menue appears where the user can select either whisper or mute someone.
	(see user requirement UR002)}{A}

\FR{The system allows to upload files in sessions and the user can sets restrictions to visibillity to other users.
    (see user requirement UR003)}{A}

\FR{Only authors of uploaded files, moderators or administrators can delete them.
    (see user requirement UR003)}{A}

\FR{The system allows to upload files out of sessions and the user can sets restrictions to visibillity to other users.  The system generates an unique identifier (URI) under which other users can reach the file.
    (see user requirement UR003)}{A}

\FR{The system is designed such that can adopt well defined plugins.
    (see user requirement UR004)}{A}

\FR{One Plugin is the conference tool. This plugin gives the possibility to handle a big conference such a lecture with the system. Therefore different views are necessary. A conference needs an audience which sees the speaker, a schedule of the conference and the slides on the screen. They can also show the speaker that there is an question. The speaker has another view. He can see how many time left and which slides comes next. Furthermore a view for an conference designer is useful. He can see a plan of the whole conference, the current activities and can edit the schedule.  At least a view for the presentation manager is used. He decides which part of presentation the audience can see. Just the slides or slides and speaker or additional information which he can add. The system shall also give the opportunity that an audience out of the conference romm can see the presentation.
		(see user requirement UR005)}{B}

\FR{The system must give the opportunity of user-registration via e-mail-address, username, and password.
    (see user requirement UR008)}{A}

\FR{The system shall give the possibility to record \gls{recordable plugins} while sessions. Therefore the user can select which plugins in a session he wants to record. 
    (see user requirement UR009)}{A}

\FR{Every registered user can choose whether he/she can be recorded or not. For an unregistrered user this option shall be set to no recordings.
    (see user requirement UR021)}{A}

\FR{The system denies the recording of a plugin if in the session is a user who forbids the recording
    (see user requirement UR021)}{A}

\FR{The systems stores records of differnt plugins in several files.
    (see user requirement UR009)}{A}

\FR{For an whiteboard recording the system stores every step by drawing - not only the end-result.
    (see user requirement UR009)}{A}

\FR{For an text chat the chat history will be stored as plain/text since the recording started.
    (see user requirement UR009)}{A}

\FR{For a video recording a video file will be saved since the recording is started.
    (see user requirement UR009)}{A}

\FR{For a audio recording the audio file will be saved since the recording started.
    (see user requirement UR009)}{A}

\FR{The system have to include the plugin of a digital whiteboard on which users can draw geometrical objects or write text. They can use it on their own or in sessions.
    (see user requirement UR010)}{A}

\FR{The system have to give the opportunity to include a whiteboard to a session, that other users in the session can hava a look at the whiteboard or edit the whiteboard too.
    (see user requirement UR010)}{A}

\FR{By including the whiteboard to a session the user has to decide whether the members of the session can only see the whiteboard or even edit things.
    (see user requirement UR010)}{A}

\FR{The system shall give a user the possibility to have more than one session.
    (see user requirement UR013)}{A}

\FR{The system denies a new session if with a new session the quality of the other sessions falls under a well defined value. For video chat this border lies by X kB and by voice chat by Y kB.
    (see user requirement UR013)}{A}

\FR{The system devides between three different arts of plugins. The first are high traffic plugins, the secont are medium traffic plugins and the third are less traffic plugins. So the system gives 10 percent of speed to the less traffic plugins, 30 percent to the medium traffic plugins, and 60 percent of speed to the high traffic plugins.
    (see user requirement UR013)}{A}

\FR{Every plugin can be marked (but only one at a time) as priorised, so that this plugin gets the most speed of the internet connection and the other plugins only gets the minimal value.
    (see user requirement UR013)}{A}

\FR{The system can deny new session who run over the server or
      a file upload if he has overload.
    (see user requirement UR013)}{A}

\FR{In a session a user can include a arbitrary number of plugins at the same time. 
    (see user requirement UR014)}{A}

\FR{The system have to give the possibillity to add every plugin to a session.
    (see user requirement UR028)}{A}

\FR{The User has the possibility to include every opened dock on his flex dock in a session.
    (see user requirement UR012, UR014)}{A}

\FR{Every Plugin can have its own window inside the multi window view.
    (see user requirement UR015)}{A}

\FR{Sessions or even parts of sessions (Plugins that are storable) can be stored and reloaded from the session admin.
		(see user requirement UR027)}{A}

\FR{When a user includes a plugin to a session, the user has to decide which rights the specific users in the session has on the plugin.
		(see user requirement UR023)}{A}

\FR{While the session the user who added a plugin to the session (only this user) can change the permissions of the other users.
		(see user requirement UR023)}{A}

\FR{The system supports registered and unregistered users. 
		(see user requirement UR023)}{A}

\FR{Registered shall be able to store their settings made at previous logins, sessions and able to record plugins. Unregistered user always start with default settings and changes they made
are only valid for their current session. 
		(see user requirement UR020)}{A}

\FR{The system provides a Weka plugin where the user must
      specify a dataset in arff format (which have to be located on
      the server where the application is running), parameters and the
      algorithm which shall be used. (see user requirement UR005)}{A}

\FR{The system provides an OpenTox plugin which implements the OpenTox \gls{API}. (see user requirement UR005)}{A}

\FR{When a user shares his/her screen all rights of other users are temporarily set to readonly. (see user requirement UR006)}{A}

\FR{The system includes a version-control-system to provide sharing and editing documents. The user can select and edit the latest version but has the opportunity to open older revision in a readonly mode. When changing a file a new revision is created and uploaded to the server. (see user requirement UR007)}{A}

\FR{When two users trying to work simultanously on the same file there are two cases. \\In the case that both users are in the same session they can work simultanously. \\In the other case the user who wants to open the file secondly opens the file in an readonly mode. (see user requirement UR007)}{A}

\FR{When a user logs in he/she has got the possibiltiy to add
      other registered users to its adressbook. A user can be added in
      3 ways.\\1. When a user is in a conference he/she can click on
      an other username and select "Add to adressbook".\\2. On every
      users profile there is an option to add these contact to the
      adressbook.\\3. When a user invites someone and this person registers they both appear in each others adressbook.\\Of course users can be removed from an adressbook. There is also the possibillity to organize the contacts in groups. (see user requirement UR011)}{A}

\FR{With the addressbook it is possible to invite multiple users in a group with one click to a conference. (see user requirement UR011)}{A}

\FR{The system provides a \gls{Flexdock} GUI where the user can freely drag and resize windows. The flexdock is designed that every plugin can have its own dock. (see user requirement UR012, UR015)}{A}

\FR{Users can customize the appearance in the user settings. (see user requirement UR016)}{A}

\FR{The system makes use of \gls{ICE}. ICE delivers techniques to establish connections to clients even they are behind a firewall or router.  (see user requirement UR017)}{A}

\FR{The system supports the following
      browsers:\\1. Chrome/Chromium X\\2. Firefox Y.\\ 3. Opera (see user requirement UR018)}{A}

\FR{The system supports at least one of the available browsers
      for the following mobile operating systems:\\1. Android
      2.1+\\2. iOS 5.0+\\No additional app (except perhaps browser) is needed. (see user requirement UR018, UR021, UR024)}{A}

\FR{The well defined plugin api ensures that every plugin offers ways to handle different rights and views. (see user requirement UR023)}{A}

\FR{The system provides a settings area where users can view and change their current settings (background colour, font-size, font, font-colour and if its allowed to record the user) made to the theme and the plugins. (see user requirement UR016)}{A}

\FR{The only things which are needed to use the software are an internet-connection and a browser which deals with HTML5 and the webRTC. (see user requirement UR016)}{A}

\FR{The system makes the user who first invites another member to a plugin (so he starts a session) to the session admin. (see user requirement UR032)}{A}

\FR{The system allows the session admin to give permissions to the session members. These permissions are the right to add plugins or kick and add other users. (see user requirement UR032)}{A}

\FR{Every user registered or not is able to invite other people. The only thing he/she needs is a valid email adress to where the invitationlink is sent. (see user requirement UR031)}{A}

\FR{The system provides an administration area where the system admin can trigger manual backups and has database access where he/she can update the period between the garbage collection. (see user requirement UR029)}{A}