\FR{Das System kann über \gls{REST} und \gls{HTTPS} bedient werden.	(siehe user requirement UR001)}{A}


\FR{Bei gleichen Eingaben sollen gleiche Ergebnisse bzw. Vorhersagen generiert werden. (siehe user requirement UR002)}{A}

\FR{Die Funktionsfähigkeit der nicht vorimplementierten Algorithmen muss durch den \gls{admin} gewährleistet werden. (siehe user requirement UR002)}{A}

\FR{Bei gleichen Eingaben sollen gleiche Ergebnisse bzw. Vorhersagen generiert werden. (siehe user requirement UR002)}{A}



\FR{Das System soll das Hochladen von Daten ermöglichen, sowie das Festsetzen von Berechtigungen für Selbige. (siehe user requirement UR003)}{A}

\FR{Eine Validierung der Daten ist nicht notwendig. (siehe user requirement UR003)}{A}

\FR{Das System generiert einen eindeutige \gls{URI} für jede Datei, über die der Nutzer auf jene zugreifen kann. (siehe user requirement UR003)}{A}


\FR{Die Datensätze sowie die Auswahl des Algorithmus erfolgt über eine tabellarisch angeordnete Eingabemaske. (siehe user requirement UR004)}{A}

\FR{Die eingegebenen Daten müssen nicht vom System auf ihre Richtigkeit geprüft werden. (siehe user requirement UR004)}{A}


\FR{Das System soll plattformübergreifend einsetzbar sein. (siehe user requirement UR005)}{A}


\FR{Das System soll Java \glspl{Interface} zur Erstellung von \glspl{Plugin} zur Verfügung stellen. Das Interface soll festlegen welche Datensätze das Plugin verarbeiten kann und welche Rückgabewerte zu erwarten sind. (siehe user requirement UR006)}{A}





\FR{The system must give the opportunity of user-registration via e-mail-address, username, and password.
    (siehe user requirement UR008)}{A}

\FR{The system shall give the possibility to record \gls{recordable plugins} while sessions. Therefore the user can select which plugins in a session he wants to record. 
    (siehe user requirement UR009)}{A}

\FR{Every registered user can choose whether he/she can be recorded or not. For an unregistrered user this option shall be set to no recordings.
    (siehe user requirement UR021)}{A}

\FR{The system denies the recording of a plugin if in the session is a user who forbids the recording
    (siehe user requirement UR021)}{A}

\FR{The systems stores records of differnt plugins in several files.
    (siehe user requirement UR009)}{A}

\FR{For an whiteboard recording the system stores every step by drawing - not only the end-result.
    (siehe user requirement UR009)}{A}

\FR{For an text chat the chat history will be stored as plain/text since the recording started.
    (siehe user requirement UR009)}{A}

\FR{For a video recording a video file will be saved since the recording is started.
    (siehe user requirement UR009)}{A}

\FR{For a audio recording the audio file will be saved since the recording started.
    (siehe user requirement UR009)}{A}

\FR{The system have to include the plugin of a digital whiteboard on which users can draw geometrical objects or write text. They can use it on their own or in sessions.
    (siehe user requirement UR010)}{A}

\FR{The system have to give the opportunity to include a whiteboard to a session, that other users in the session can hava a look at the whiteboard or edit the whiteboard too.
    (siehe user requirement UR010)}{A}

\FR{By including the whiteboard to a session the user has to decide whether the members of the session can only see the whiteboard or even edit things.
    (siehe user requirement UR010)}{A}

\FR{The system shall give a user the possibility to have more than one session.
    (siehe user requirement UR013)}{A}

\FR{The system denies a new session if with a new session the quality of the other sessions falls under a well defined value. For video chat this border lies by X kB and by voice chat by Y kB.
    (siehe user requirement UR013)}{A}

\FR{The system devides between three different arts of plugins. The first are high traffic plugins, the secont are medium traffic plugins and the third are less traffic plugins. So the system gives 10 percent of speed to the less traffic plugins, 30 percent to the medium traffic plugins, and 60 percent of speed to the high traffic plugins.
    (siehe user requirement UR013)}{A}

\FR{Every plugin can be marked (but only one at a time) as priorised, so that this plugin gets the most speed of the internet connection and the other plugins only gets the minimal value.
    (siehe user requirement UR013)}{A}

\FR{The system can deny new session who run over the server or
      a file upload if he has overload.
    (siehe user requirement UR013)}{A}

\FR{In a session a user can include a arbitrary number of plugins at the same time. 
    (siehe user requirement UR014)}{A}

\FR{The system have to give the possibillity to add every plugin to a session.
    (siehe user requirement UR028)}{A}

\FR{The User has the possibility to include every opened dock on his flex dock in a session.
    (siehe user requirement UR012, UR014)}{A}

\FR{Every Plugin can have its own window inside the multi window view.
    (siehe user requirement UR015)}{A}

\FR{Sessions or even parts of sessions (Plugins that are storable) can be stored and reloaded from the session admin.
		(siehe user requirement UR027)}{A}

\FR{When a user includes a plugin to a session, the user has to decide which rights the specific users in the session has on the plugin.
		(siehe user requirement UR023)}{A}

\FR{While the session the user who added a plugin to the session (only this user) can change the permissions of the other users.
		(siehe user requirement UR023)}{A}

\FR{The system supports registered and unregistered users. 
		(siehe user requirement UR023)}{A}

\FR{Registered shall be able to store their settings made at previous logins, sessions and able to record plugins. Unregistered user always start with default settings and changes they made
are only valid for their current session. 
		(siehe user requirement UR020)}{A}

\FR{The system provides a Weka plugin where the user must
      specify a dataset in arff format (which have to be located on
      the server where the application is running), parameters and the
      algorithm which shall be used. (siehe user requirement UR005)}{A}

\FR{The system provides an OpenTox plugin which implements the OpenTox \gls{API}. (siehe user requirement UR005)}{A}

\FR{When a user shares his/her screen all rights of other users are temporarily set to readonly. (siehe user requirement UR006)}{A}

\FR{The system includes a version-control-system to provide sharing and editing documents. The user can select and edit the latest version but has the opportunity to open older revision in a readonly mode. When changing a file a new revision is created and uploaded to the server. (siehe user requirement UR007)}{A}

\FR{When two users trying to work simultanously on the same file there are two cases. \\In the case that both users are in the same session they can work simultanously. \\In the other case the user who wants to open the file secondly opens the file in an readonly mode. (siehe user requirement UR007)}{A}

\FR{When a user logs in he/she has got the possibiltiy to add
      other registered users to its adressbook. A user can be added in
      3 ways.\\1. When a user is in a conference he/she can click on
      an other username and select "Add to adressbook".\\2. On every
      users profile there is an option to add these contact to the
      adressbook.\\3. When a user invites someone and this person registers they both appear in each others adressbook.\\Of course users can be removed from an adressbook. There is also the possibillity to organize the contacts in groups. (siehe user requirement UR011)}{A}

\FR{With the addressbook it is possible to invite multiple users in a group with one click to a conference. (siehe user requirement UR011)}{A}

\FR{The system provides a \gls{Flexdock} GUI where the user can freely drag and resize windows. The flexdock is designed that every plugin can have its own dock. (siehe user requirement UR012, UR015)}{A}

\FR{Users can customize the appearance in the user settings. (siehe user requirement UR016)}{A}

\FR{The system makes use of \gls{ICE}. ICE delivers techniques to establish connections to clients even they are behind a firewall or router.  (siehe user requirement UR017)}{A}

\FR{The system supports the following
      browsers:\\1. Chrome/Chromium X\\2. Firefox Y.\\ 3. Opera (siehe user requirement UR018)}{A}

\FR{The system supports at least one of the available browsers
      for the following mobile operating systems:\\1. Android
      2.1+\\2. iOS 5.0+\\No additional app (except perhaps browser) is needed. (siehe user requirement UR018, UR021, UR024)}{A}

\FR{The well defined plugin api ensures that every plugin offers ways to handle different rights and views. (siehe user requirement UR023)}{A}

\FR{The system provides a settings area where users can view and change their current settings (background colour, font-size, font, font-colour and if its allowed to record the user) made to the theme and the plugins. (siehe user requirement UR016)}{A}

\FR{The only things which are needed to use the software are an internet-connection and a browser which deals with HTML5 and the webRTC. (siehe user requirement UR016)}{A}

\FR{The system makes the user who first invites another member to a plugin (so he starts a session) to the session admin. (siehe user requirement UR032)}{A}

\FR{The system allows the session admin to give permissions to the session members. These permissions are the right to add plugins or kick and add other users. (siehe user requirement UR032)}{A}

\FR{Every user registered or not is able to invite other people. The only thing he/she needs is a valid email adress to where the invitationlink is sent. (siehe user requirement UR031)}{A}

\FR{The system provides an administration area where the system admin can trigger manual backups and has database access where he/she can update the period between the garbage collection. (siehe user requirement UR029)}{A}