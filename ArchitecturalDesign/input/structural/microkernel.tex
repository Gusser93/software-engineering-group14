Das System stellt seine Funktionalität über die WEKA-Libary zur Verfügung. Damit diese arbeiten kann, werden Algorithmen benötigt. 
Um eine dynamische Ergänzung der Algorithmen zu ermöglichen wird WEKA als Mikrokernel implementiert. 
So können die Algorithmen als interne Server, wenn benötigt, geladen werden und neue Algorithmen können hinzugefügt werden, ohne das der WEKA-Quellcode bearbeitet werden muss. 
Alle Anfragen an WEKA laufen dabei über eine Datenschnittstelle, welche verschiedene Funktionalitäten für den Client bereitstellt.
So kann die Datenschnittstelle zurückgeben, welche Algorithmen von einem bestimmten Datensatz unterstützt werden oder ob ein zu erstellendes Modell bereits in der Datenbank vorhanden ist. 
WEKA übernimmt die Berechnung eines Modells und die Auswertung eines Datensatzes (bzw. eines Algorithmus). 
Der Web-Server übernimmt die Rolle eines Adapters, welcher die REST-Anfragen des Clients auswertet und weitere Instruktionen einleitet.\\
\begin{figure}[h]
\centering
	\vspace{-5pt}
\includegraphics[width=0.7\linewidth]{Grafik/Diagramm/Microkernel}
\caption[Microkernel-Klassen]{Microkernel-Pattern mit WEKA}
\label{fig:Microkernel}
\end{figure}

\subsubsection{Was spricht für den Microkernel?}
Das System soll in der Lage sein stetig mit neuen Algorithmen versorgt zu werden. Das Microkernel-Pattern erlaubt dies, ohne im späteren Verlauf oder gar für jeden Algorithmus den Quellcode verändern zu müssen. Dies erlaubt eine große Flexibilität hinsichtlich der Möglichkeit verschiedene Algorithmen zu verwenden. Ein weiterer Aspekt ist die Trennung von WEKA und den Algorithmen. Dies verringert das negative Beeinflussen beider Parteien untereinander. Des Weiteren erlaubt es ein einfacheres ändern (z.B. Updaten, erweitern der Funktionalität) der Komponenten. In Anbetracht der Vorteile, sind die Performance-Einbußen durch das Pattern nicht weiter relevant.