\subsubsection{Was spricht gegen Pipe-and-Filter-Pattern?}
Der große Vorteil des Pipe-and-Filter-Pattern besteht darin kontinuierliche Datenströme, mit Hilfe verschiedener Filterbausteine, asynchron und meist unabhängig von einander zu verarbeiten. Da das System allerdings hauptsächlich auf einfachen Dateiaustausch und kurze Anfragen an den Server basiert, reicht es vollkommen aus eine normale https Verbindung zu verwenden ohne eigene Filterströme zwischenzuschalten. Dieses Pattern wäre für die Kommunikation zwischen Client und Server zu mächtig und würde lediglich zu unnötigen Leistungseinbußen führen.