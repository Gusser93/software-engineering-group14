\documentclass{beamer}
\usepackage[ngerman]{babel}			%Deutsche Umlaute und Umbrüche
\usepackage[utf8]{inputenc}			%utf8 Kodierung
\usepackage{amsmath, amsfonts, amssymb} %Mathepackete schaden nie
\usetheme{Dresden}
\usecolortheme{beaver}
\usefonttheme{professionalfonts}
\setbeamertemplate{footline}[frame number]
\title[WWWEKA]{World Wide WEKA}
\subtitle{Requirement Document}
\author[C. Heckmann\and C. Stricker\and D. Klopp\and M. Vieth]{Christian Heckmann\and Christian Stricker\and David Klopp\and Markus Vieth}
\date[01.01.1111]{01. Januar 1111}
\subject{Software engineering}
%Link zum Beamer-Userguid: ftp://ftp.dante.de/tex-archive/macros/latex/contrib/beamer/doc/beameruserguide.pdf
\begin{document}
	
	\frame{
		\titlepage
	}
	
	\frame[label=IV]{
		\frametitle{Inhaltsverzeichnis}
		\tableofcontents
		[pausesections]
	}
	\section[Beispiele]{Beispiel Folien}
	%Testfolien
	\subsection[Listen]{Beispiele für Listen}
	\begin{frame}[<+->][t]{Listen}{Untertitel}
		\begin{itemize}
			\item eins
			\item zwei
		\end{itemize}
		\begin{enumerate}
			\item eins
			\item zwei
		\end{enumerate}
		\begin{description}
			\item[erstens] 1.
			\item[zweitens] 2.
		\end{description}
	\end{frame}
	\subsection[Blöcke]{Beispiele für Blöcke}
	\subsubsection[Nomal]{Beispiele für normale Blöcke}
	\begin{frame}{Blöcke}
		
		\begin{block}{Titel eines einfachen Blocks}
			Text eines einfachen Blocks.
		\end{block}
		\begin{exampleblock}{Titel eines Beispiel-Blocks}
			Text eines Beispiel-Blocks.
		\end{exampleblock}
		\begin{alertblock}{Titel eines Alarm-Blocks}
			Text eines Alarm-Blocks.
		\end{alertblock}
	\end{frame}
	\subsubsection[Mathe]{Beispiele für Mathe-Blöcke}
	\begin{frame}{Mathe-Blöcke}
		\begin{proof}
			Ein Beweis
		\end{proof}
		\begin{definition}
			Eine Definition
		\end{definition}
		\begin{lemma}[Lemma titel -- toll]
			ein Lemma
		\end{lemma}
		\begin{theorem}[M -- Nach Markus]
			Ein Theorem
		\end{theorem}
	\end{frame}
	\subsection[Animation]{Beispiel für Animationen}
	\begin{frame}
		\begin{itemize}
			\item Einleitung
			\item<2-> daher
			\item<2-|alert@3-> aber Achtung!
			\item<3-> also so und so
			\item<4-> Schluss
		\end{itemize}
	\end{frame}
	
\end{document}