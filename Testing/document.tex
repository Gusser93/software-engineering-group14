\documentclass{beamer}
\usepackage[ngerman]{babel}			%Deutsche Umlaute und Umbrüche
\usepackage[utf8]{inputenc}			%utf8 Kodierung
\usepackage{amsmath, amsfonts, amssymb, ulem, epstopdf} %Mathepackete schaden nie
\usetheme{Dresden}
\usecolortheme{beaver}
\usefonttheme{professionalfonts}
\title[Tofuml]{Tofuml}
\subtitle{Testcase}
\author[C. Stricker\and A. Klein\and D. Klopp\and M. Vieth]{Christian Stricker\and Alisha Klein\and David Klopp\and Markus Vieth}
\date[\today]{\today}
\subject{Software engineering}
\newcommand{\btVFill}{\vskip0pt plus 1filll}
\setcounter{tocdepth}{2}
%Link zum Beamer-Userguid: ftp://ftp.dante.de/tex-archive/macros/latex/contrib/beamer/doc/beameruserguide.pdf
\begin{document}
	
	\frame{
		\titlepage
	}
	\setbeamertemplate{footline}[frame number]
	
	\frame[label=IV]{
		\frametitle{Inhaltsverzeichnis}
		\tableofcontents
		[pausesections]
	}
	
	\section{Test-Kategorien}
	\subsection{Development Testing}
	\begin{frame}%{Development Testing}
		\begin{itemize}
		\item Zum Entdecken von Fehlern
		\item Vor bzw. Parallel zur Entwicklung
		\item Unterteilt in Komponententest, Integrationstest, Systemtest und Abnahmetest
	\end{itemize}
	\end{frame}
	
	\begin{frame}{Komponententest}
		\begin{itemize}
			\item auch Unittest genannt
			\item Testet einzelne Module
			\item Funktionalität abgrenzbarer Teile der Software
			
		\end{itemize}
	\end{frame}	
	\begin{frame}{Integrationstest}
		\begin{itemize}
			\item Interaktionstest genannt
			\item Zusammenarbeit von abhängigen Komponenten
			\item Testet Schnittstellen	
			\item Nachweis von Korrekten Ergebnissen über komplette Abläufe des Programms
		\end{itemize}
	\end{frame}
	\begin{frame}{Systemtest}
	\begin{itemize}
		\item Testet gesamtes System gegen gesamte Anforderung (Function/non-Function-Requirements)	
		\item Testen mit Testdaten
		\item Kundenbenutzungssimulation
	\end{itemize}
	\end{frame}	
	\begin{frame}{Abnahmetest}
		\begin{itemize}
			\item Getestet vom Kunden
			\item Positiver Test für rechtswirksame Übernahme
			\item Test auf Produktionsumgebung mit meist echten Testdaten
			\item Blackboxverfahren
		\end{itemize}
	\end{frame}
	\subsection{Release testing}
	\begin{frame}
		\begin{itemize}
			\item Nach der Entwicklung, aber vor dem Release
			\item Verschiedene Gruppen testen die Software
			\item Das System ist fertig
			\item Validierung des Systems
			\item Beinhaltet Performance und Stress Tests
		\end{itemize}
	\end{frame}
	\subsection{User testing}
	\begin{frame}
		\begin{itemize}
			\item Nutzer testen das System
			\item meist unterteilt in Alpha- und Beta-Tests
			\item bei Custom Software auch Acceptance test
		\end{itemize}
	\end{frame}
	\section{Tools}
	\subsection{Release testing}
	\begin{frame}
		\begin{itemize}
			\item Testfälle
			\item Szenarien
			\item Benchmark
		\end{itemize}
	\end{frame}
	\subsection{User testing}
	\begin{frame}
		\begin{itemize}
			\item Feedbacksystem
			\item Forum
			\item Ticketsystem
			\item Bugtracker (Bugzilla)
		\end{itemize}
	\end{frame}
	\section{How to test}
	\subsection{Release testing}
	\begin{frame}
		\begin{itemize}
			\item QC teams
			\item Entwickeln von Einsatzfällen/Beispielen
			\item Abgleich mit User-/Systemrequirements
		\end{itemize}
	\end{frame}
	\subsection{User testing}
	\begin{frame}
		\begin{itemize}
			\item geschlossene Alpha Testgruppe mit persönlichem Feedbacksystem
			\item offene Beta Testphase mit Foren, Ticketsystem und Bug-Tracker
		\end{itemize}
	\end{frame}
		
\end{document}